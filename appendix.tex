\onecolumn
\appendix
Note that the pseudocode  below describes how Max-Min and Min-Min work but does not include implementation details in regards to how
a task is executed or how files are sent. 

\begin{algorithm}[H]
  \renewcommand{\thealgorithm}{}
  \caption{Max-Min}
  \begin{algorithmic}
      \STATE $pending\_tasks = application.getTasks()$
      \WHILE{$pending\_tasks.size \neq 0$}
        \IF{$canScheduleTasks()$}
          \STATE $ECTS = [|pending\_tasks|][|workers|]$\COMMENT{estimated completion times}
          \FOR{$t \in pending\_tasks$}
            \FOR{$w \in workers$}
              \STATE $ect = estimate\_completion\_time(t, w)$
              \STATE $ECTS[t][w] = ect$
            \ENDFOR
          \ENDFOR

          \STATE $target\_task = \set{t \given \max\limits_{\forall t \in pending\_tasks} (\min\limits_{\forall w
          \in workers}(ECT[t][w]))}$
          \STATE $target\_worker = \set{w \given \max\limits_{\forall w \in workers}(ECT[target\_task][w])}$
          \STATE $assign(target\_task, target\_worker)$

        \ENDIF
      \ENDWHILE
  \end{algorithmic}
\end{algorithm}

\begin{algorithm}[H]
  \renewcommand{\thealgorithm}{}
  \caption{Min-Min}
  \begin{algorithmic}
      \STATE $pending\_tasks = application.getTasks()$
      \WHILE{$pending\_tasks.size \neq 0$}
        \IF{$canScheduleTasks()$}
          \STATE $ECTS = [|pending\_tasks|][|workers|]$\COMMENT{estimated completion times}
          \FOR{$t \in pending\_tasks$}
            \FOR{$w \in workers$}
              \STATE $ect = estimate\_completion\_time(t, w)$
              \STATE $ECTS[t][w] = ect$
            \ENDFOR
          \ENDFOR

          \STATE $target\_task, = \set{t \given \min\limits_{\forall t \in pending\_tasks} (\min\limits_{\forall w
          \in workers}(ECT[t][w]))}$
          \STATE $target\_worker = \set{w \given \min\limits_{\forall w \in workers}(ECT[target\_task][w])}$
          \STATE $assign(target\_task, target\_worker)$

        \ENDIF
      \ENDWHILE
  \end{algorithmic}
\end{algorithm}
