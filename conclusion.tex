\section{conclusion}
\label{sec:conclusion}
Based on the results in this study, we can conclude that MinMin and MaxMin
may perform well when given random applications and master worker
infrastructures. Additionally, given enough iterations, the GA could
produce a schedule that performs just as well, or better than these
list scheduling heuristics. In practice, online and offine heuristics
generally aren't compared, however in the case of scheduling parameter
sweep applications, this can be okay as real world applications may run
for days or months while the evolutionary algorithm may run for a fraction
of that time thanks to simulation.

Both the GA and the heuristics have their advantages and disadvangtages. First,
the genotype representation for this specific problem scales in the size of
the number of files used in the application along with the number of worker
nodes. Real world parameter sweep applications have a number of files and tasks
orders of magnitude greater than the number of worker nodes available. Using
this specific GA in a real world scenario requires a genotype representation
that could be thousands of times larger than what was used in this study.
If the length of the genotype is $N$, then we end up with a search space
that is $N!$. Thus scaling GAs for larger problems similar to this have
been pointed out to be a challenge \cite{wu-incremental-genetic-04}. The search
space in this study is $500!$ and the GA came close to the heuristics after
only 30000 fitness evaluations, a minute portion of the total search space.
