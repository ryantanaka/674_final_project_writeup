%% bare_conf.tex
%% V1.4b
%% 2015/08/26
%% by Michael Shell
%% See:
%% http://www.michaelshell.org/
%% for current contact information.
%%
%% This is a skeleton file demonstrating the use of IEEEtran.cls
%% (requires IEEEtran.cls version 1.8b or later) with an IEEE
%% conference paper.
%%
%% Support sites:
%% http://www.michaelshell.org/tex/ieeetran/
%% http://www.ctan.org/pkg/ieeetran
%% and
%% http://www.ieee.org/

%%*************************************************************************
%% Legal Notice:
%% This code is offered as-is without any warranty either expressed or
%% implied; without even the implied warranty of MERCHANTABILITY or
%% FITNESS FOR A PARTICULAR PURPOSE!
%% User assumes all risk.
%% In no event shall the IEEE or any contributor to this code be liable for
%% any damages or losses, including, but not limited to, incidental,
%% consequential, or any other damages, resulting from the use or misuse
%% of any information contained here.
%%
%% All comments are the opinions of their respective authors and are not
%% necessarily endorsed by the IEEE.
%%
%% This work is distributed under the LaTeX Project Public License (LPPL)
%% ( http://www.latex-project.org/ ) version 1.3, and may be freely used,
%% distributed and modified. A copy of the LPPL, version 1.3, is included
%% in the base LaTeX documentation of all distributions of LaTeX released
%% 2003/12/01 or later.
%% Retain all contribution notices and credits.
%% ** Modified files should be clearly indicated as such, including  **
%% ** renaming them and changing author support contact information. **
%%*************************************************************************


% *** Authors should verify (and, if needed, correct) their LaTeX system  ***
% *** with the testflow diagnostic prior to trusting their LaTeX platform ***
% *** with production work. The IEEE's font choices and paper sizes can   ***
% *** trigger bugs that do not appear when using other class files.       ***                          ***
% The testflow support page is at:
% http://www.michaelshell.org/tex/testflow/



\documentclass[11pt, journal]{IEEEtran}
% Some Computer Society conferences also require the compsoc mode option,
% but others use the standard conference format.
%
% If IEEEtran.cls has not been installed into the LaTeX system files,
% manually specify the path to it like:
% \documentclass[conference]{../sty/IEEEtran}





% Some very useful LaTeX packages include:
% (uncomment the ones you want to load)


% *** MISC UTILITY PACKAGES ***
%
%\usepackage{ifpdf}
% Heiko Oberdiek's ifpdf.sty is very useful if you need conditional
% compilation based on whether the output is pdf or dvi.
% usage:
% \ifpdf
%   % pdf code
% \else
%   % dvi code
% \fi
% The latest version of ifpdf.sty can be obtained from:
% http://www.ctan.org/pkg/ifpdf
% Also, note that IEEEtran.cls V1.7 and later provides a builtin
% \ifCLASSINFOpdf conditional that works the same way.
% When switching from latex to pdflatex and vice-versa, the compiler may
% have to be run twice to clear warning/error messages.






% *** CITATION PACKAGES ***
%
\usepackage{cite}
% cite.sty was written by Donald Arseneau
% V1.6 and later of IEEEtran pre-defines the format of the cite.sty package
% \cite{} output to follow that of the IEEE. Loading the cite package will
% result in citation numbers being automatically sorted and properly
% "compressed/ranged". e.g., [1], [9], [2], [7], [5], [6] without using
% cite.sty will become [1], [2], [5]--[7], [9] using cite.sty. cite.sty's
% \cite will automatically add leading space, if needed. Use cite.sty's
% noadjust option (cite.sty V3.8 and later) if you want to turn this off
% such as if a citation ever needs to be enclosed in parenthesis.
% cite.sty is already installed on most LaTeX systems. Be sure and use
% version 5.0 (2009-03-20) and later if using hyperref.sty.
% The latest version can be obtained at:
% http://www.ctan.org/pkg/cite
% The documentation is contained in the cite.sty file itself.






\usepackage[pdftex]{graphicx}
\usepackage{pdfpages}
\usepackage{float}
% *** GRAPHICS RELATED PACKAGES ***
%
\ifCLASSINFOpdf
  \usepackage[pdftex]{graphicx}
  % declare the path(s) where your graphic files are
  % \graphicspath{{../pdf/}{../jpeg/}}
  % and their extensions so you won't have to specify these with
  % every instance of \includegraphics
  % \DeclareGraphicsExtensions{.pdf,.jpeg,.png}
\else
  % or other class option (dvipsone, dvipdf, if not using dvips). graphicx
  % will default to the driver specified in the system graphics.cfg if no
  % driver is specified.
  % \usepackage[dvips]{graphicx}
  % declare the path(s) where your graphic files are
  % \graphicspath{{../eps/}}
  % and their extensions so you won't have to specify these with
  % every instance of \includegraphics
  % \DeclareGraphicsExtensions{.eps}
\fi
% graphicx was written by David Carlisle and Sebastian Rahtz. It is
% required if you want graphics, photos, etc. graphicx.sty is already
% installed on most LaTeX systems. The latest version and documentation
% can be obtained at:
% http://www.ctan.org/pkg/graphicx
% Another good source of documentation is "Using Imported Graphics in
% LaTeX2e" by Keith Reckdahl which can be found at:
% http://www.ctan.org/pkg/epslatex
%
% latex, and pdflatex in dvi mode, support graphics in encapsulated
% postscript (.eps) format. pdflatex in pdf mode supports graphics
% in .pdf, .jpeg, .png and .mps (metapost) formats. Users should ensure
% that all non-photo figures use a vector format (.eps, .pdf, .mps) and
% not a bitmapped formats (.jpeg, .png). The IEEE frowns on bitmapped formats
% which can result in "jaggedy"/blurry rendering of lines and letters as
% well as large increases in file sizes.
%
% You can find documentation about the pdfTeX application at:
% http://www.tug.org/applications/pdftex





% *** MATH PACKAGES ***
%
%\usepackage{amsmath}
% A popular package from the American Mathematical Society that provides
% many useful and powerful commands for dealing with mathematics.
%
% Note that the amsmath package sets \interdisplaylinepenalty to 10000
% thus preventing page breaks from occurring within multiline equations. Use:
%\interdisplaylinepenalty=2500
% after loading amsmath to restore such page breaks as IEEEtran.cls normally
% does. amsmath.sty is already installed on most LaTeX systems. The latest
% version and documentation can be obtained at:
% http://www.ctan.org/pkg/amsmath





% *** SPECIALIZED LIST PACKAGES ***
%
%\usepackage{algorithmic}
% algorithmic.sty was written by Peter Williams and Rogerio Brito.
% This package provides an algorithmic environment fo describing algorithms.
% You can use the algorithmic environment in-text or within a figure
% environment to provide for a floating algorithm. Do NOT use the algorithm
% floating environment provided by algorithm.sty (by the same authors) or
% algorithm2e.sty (by Christophe Fiorio) as the IEEE does not use dedicated
% algorithm float types and packages that provide these will not provide
% correct IEEE style captions. The latest version and documentation of
% algorithmic.sty can be obtained at:
% http://www.ctan.org/pkg/algorithms
% Also of interest may be the (relatively newer and more customizable)
% algorithmicx.sty package by Szasz Janos:
% http://www.ctan.org/pkg/algorithmicx




% *** ALIGNMENT PACKAGES ***
%
%\usepackage{array}
% Frank Mittelbach's and David Carlisle's array.sty patches and improves
% the standard LaTeX2e array and tabular environments to provide better
% appearance and additional user controls. As the default LaTeX2e table
% generation code is lacking to the point of almost being broken with
% respect to the quality of the end results, all users are strongly
% advised to use an enhanced (at the very least that provided by array.sty)
% set of table tools. array.sty is already installed on most systems. The
% latest version and documentation can be obtained at:
% http://www.ctan.org/pkg/array


% IEEEtran contains the IEEEeqnarray family of commands that can be used to
% generate multiline equations as well as matrices, tables, etc., of high
% quality.




% *** SUBFIGURE PACKAGES ***
%\ifCLASSOPTIONcompsoc
%  \usepackage[caption=false,font=normalsize,labelfont=sf,textfont=sf]{subfig}
%\else
%  \usepackage[caption=false,font=footnotesize]{subfig}
%\fi
% subfig.sty, written by Steven Douglas Cochran, is the modern replacement
% for subfigure.sty, the latter of which is no longer maintained and is
% incompatible with some LaTeX packages including fixltx2e. However,
% subfig.sty requires and automatically loads Axel Sommerfeldt's caption.sty
% which will override IEEEtran.cls' handling of captions and this will result
% in non-IEEE style figure/table captions. To prevent this problem, be sure
% and invoke subfig.sty's "caption=false" package option (available since
% subfig.sty version 1.3, 2005/06/28) as this is will preserve IEEEtran.cls
% handling of captions.
% Note that the Computer Society format requires a larger sans serif font
% than the serif footnote size font used in traditional IEEE formatting
% and thus the need to invoke different subfig.sty package options depending
% on whether compsoc mode has been enabled.
%
% The latest version and documentation of subfig.sty can be obtained at:
% http://www.ctan.org/pkg/subfig




% *** FLOAT PACKAGES ***
%
%\usepackage{fixltx2e}
% fixltx2e, the successor to the earlier fix2col.sty, was written by
% Frank Mittelbach and David Carlisle. This package corrects a few problems
% in the LaTeX2e kernel, the most notable of which is that in current
% LaTeX2e releases, the ordering of single and double column floats is not
% guaranteed to be preserved. Thus, an unpatched LaTeX2e can allow a
% single column figure to be placed prior to an earlier double column
% figure.
% Be aware that LaTeX2e kernels dated 2015 and later have fixltx2e.sty's
% corrections already built into the system in which case a warning will
% be issued if an attempt is made to load fixltx2e.sty as it is no longer
% needed.
% The latest version and documentation can be found at:
% http://www.ctan.org/pkg/fixltx2e


%\usepackage{stfloats}
% stfloats.sty was written by Sigitas Tolusis. This package gives LaTeX2e
% the ability to do double column floats at the bottom of the page as well
% as the top. (e.g., "\begin{figure*}[!b]" is not normally possible in
% LaTeX2e). It also provides a command:
%\fnbelowfloat
% to enable the placement of footnotes below bottom floats (the standard
% LaTeX2e kernel puts them above bottom floats). This is an invasive package
% which rewrites many portions of the LaTeX2e float routines. It may not work
% with other packages that modify the LaTeX2e float routines. The latest
% version and documentation can be obtained at:
% http://www.ctan.org/pkg/stfloats
% Do not use the stfloats baselinefloat ability as the IEEE does not allow
% \baselineskip to stretch. Authors submitting work to the IEEE should note
% that the IEEE rarely uses double column equations and that authors should try
% to avoid such use. Do not be tempted to use the cuted.sty or midfloat.sty
% packages (also by Sigitas Tolusis) as the IEEE does not format its papers in
% such ways.
% Do not attempt to use stfloats with fixltx2e as they are incompatible.
% Instead, use Morten Hogholm'a dblfloatfix which combines the features
% of both fixltx2e and stfloats:
%
% \usepackage{dblfloatfix}
% The latest version can be found at:
% http://www.ctan.org/pkg/dblfloatfix




% *** PDF, URL AND HYPERLINK PACKAGES ***
%
%\usepackage{url}
% url.sty was written by Donald Arseneau. It provides better support for
% handling and breaking URLs. url.sty is already installed on most LaTeX
% systems. The latest version and documentation can be obtained at:
% http://www.ctan.org/pkg/url
% Basically, \url{my_url_here}.




% *** Do not adjust lengths that control margins, column widths, etc. ***
% *** Do not use packages that alter fonts (such as pslatex).         ***
% There should be no need to do such things with IEEEtran.cls V1.6 and later.
% (Unless specifically asked to do so by the journal or conference you plan
% to submit to, of course. )


% correct bad hyphenation here
\hyphenation{op-tical net-works semi-conduc-tor}


\begin{document}
\setlength{\parskip}{0pt}
%
% paper title
% Titles are generally capitalized except for words such as a, an, and, as,
% at, but, by, for, in, nor, of, on, or, the, to and up, which are usually
% not capitalized unless they are the first or last word of the title.
% Linebreaks \\ can be used within to get better formatting as desired.
% Do not put math or special symbols in the title.
%\title{Bare Demo of IEEEtran.cls\\ for IEEE Conferences}

%\title{Exploration Into Parallel Steganographic Encryption}
\title{Scheduling Parameter Sweep
Applications on Heterogeneous Compute Resources}

% author names and affiliations
% use a multiple column layout for up to three different
% affiliations
\author{
\IEEEauthorblockN{Ryan Tanaka}
\IEEEauthorblockA{Dept. of Information and Computer Science \\
University of Hawaii\\
Honoulu HI, 96822\\
Email: ryant@hawaii.edu}
}

%\and
%\IEEEauthorblockN{Nobody Else\\ and Just Us}
%\IEEEauthorblockA{Starfleet Academy\\
%San Francisco, California 96678--2391\\
%Telephone: (800) 555--1212\\
%Fax: (888) 555--1212}}

% conference papers do not typically use \thanks and this command
% is locked out in conference mode. If really needed, such as for
% the acknowledgment of grants, issue a \IEEEoverridecommandlockouts
% after \documentclass

% for over three affiliations, or if they all won't fit within the width
% of the page, use this alternative format:
%
%\author{\IEEEauthorblockN{Michael Shell\IEEEauthorrefmark{1},
%Homer Simpson\IEEEauthorrefmark{2},
%James Kirk\IEEEauthorrefmark{3},
%Montgomery Scott\IEEEauthorrefmark{3} and
%Eldon Tyrell\IEEEauthorrefmark{4}}
%\IEEEauthorblockA{\IEEEauthorrefmark{1}School of Electrical and Computer Engineering\\
%Georgia Institute of Technology,
%Atlanta, Georgia 30332--0250\\ Email: see http://www.michaelshell.org/contact.html}
%\IEEEauthorblockA{\IEEEauthorrefmark{2}Twentieth Century Fox, Springfield, USA\\
%Email: homer@thesimpsons.com}
%\IEEEauthorblockA{\IEEEauthorrefmark{3}Starfleet Academy, San Francisco, California 96678-2391\\
%Telephone: (800) 555--1212, Fax: (888) 555--1212}
%\IEEEauthorblockA{\IEEEauthorrefmark{4}Tyrell Inc., 123 Replicant Street, Los Angeles, California 90210--4321}}

% use for special paper notices
%\IEEEspecialpapernotice{(Invited Paper)}

% make the title area
\maketitle

\renewcommand\IEEEkeywordsname{Keywords}

\section{abstract}

% As a general rule, do not put math, special symbols or citations

\begin{abstract}

The optimization problem of scheduling tasks onto heterogenous resources in
distributed computing environments has been shown to be an NP-complete problem
in some cases and is the subject of ongoing research in the field of
distributed computing. A number of dynamic and static task-scheduling algorithms
have been proposed to tackle this problem. Of the static based scheduling
algorithms, there exist heuristic and random based search methods, all of which
can provide ideal task schedules in different application and hardware
configurations. The problem of accurately benchmarking these algorithms on
varying hardware configurations has proved to be time consuming for researchers,
and as a result, highly optimized simulation software has been developed over
the years to aid in the advancement of high performance and distributed
computing research. We first develop a genetic algorithm for the paremeter sweep
scheduling problem that uses that uses WRENCH, a workflow management
simulation framework, as a means of individual fitness evaluations. Then we
evaluate the performance of the schedule output by the genetic algorithm against
schedules produced by the Max-Min and Min-Min heuristics. Findings show that,
the heuristics, although simple to implement, provide good scheduling performance.
Additionally, the genetic algorithm's schedule performance approaches that of the
heuristics. However, 150 generations with a population size of 200 is not enough
to match the performance of those heuristics as movement through the search
space was slow, likely a result of the genotype implementation.


\end{abstract}


\begin{IEEEkeywords}
\\Scheduling, Distributed Computing, Workflow Management Systems, Genetic Algorithm
\end{IEEEkeywords}

\section{Introduction}

Distributed computing environments have become the platform of choice for
executing large scale scientific applications because they afford researchers
the ability to execute such applications in a fraction of the amount of time it
would take  to execute on a single machine. A "fraction of time" in this context
could mean an application makespan of months instead of a year assuming hardware
resources are utilized in a clever manner.  This paper focuses on one specific
type of scientific application known as a \textit{parameter sweep application},
where the entire application is composed of independent tasks which can require
a number of input files  \cite{Casanova-param-sweep-00},
\cite{Casanova-apples-param-sweep-00}. Tasks can require a varying amounts of
computation to complete and each task may use one or more files also used by
other tasks within the application. In order to execute a parameter sweep
application on a set of distributed computing resources, one must specify a
schedule with the following information: 1. when and where to send what file, 2.
when and where to execute what task. This decision making/scheduling process has
been proven to be NP-complete in many cases
\cite{Giersch-task-sharing-files-04}. Heuristic based, list scheduling
approaches such as \textit{MaxMin, MinMin} are some efficient and effective ways
to schedule file transfers and assign tasks to compute hosts
\cite{Casanova-param-sweep-00}, \cite{Casanova-apples-param-sweep-00},
\cite{Giersch-task-sharing-files-04}. Previous research has suggested that using
genetic algorithms as a viable meta-heuristic for the task scheduling problem
\cite{wang-task-matching-97} , \cite{wu-incremental-genetic-04} where inter task
dependencies exist. The purpose of this project is to benchmark a schedule
created by a genetic algorithm (GA) and compare the schedule's performance with
the afformentioned \textit{MaxMin and MinMin} heuristics. To accomplish this we
use WRENCH, a workflow management simulation framework
\cite{casanova-works-2018}, to simulate a parameter sweep application, the
cyberinfrastructure it is to be executed on, and the scheduling logic necessary
to map portions of the application onto compute and storage resources.

This paper is organized as follows. Section~\ref{sec:background} begins by
formally defining the problem of scheduling parameter sweep applications on
heterogenous resources. Then the MaxMin and MinMin heuristics are described.
Section~\ref{sec:genetic_algorithm} covers the implementation details of
the genetic algorithm used. Section~\ref{sec:experimental_details} describes
the 


\section{background} \label{sec:background}

\subsection{Parameter Sweep Applications}

Scientific applications are often represented as directed acyclic graphs (DAGs)
where nodes represent computational tasks and input/output files. Links
generally will start at a file and extend to a task or vice versa. A link from a
file to a task denotes that the task requires the file as an input. Conversely,
a link from a task to a file signifies that the task creates that file as its
output. Here we focus on one such scientific application known as a parameter
sweep, where there are an independent set of files and tasks *Figure1. These
files have links to tasks, thus forming a bipartite graph. Depending on the
application, this bipartite graph may be arranged in a number of ways. For
example, consider an application with 10 files and 20 tasks. And say, each file
is used by a pair of tasks. This type of application may be easily parallelized
as the graph can be partitioned into sections. However, in the case of irregular
applications, the dependencies between files and tasks can make it difficult to
effectively schedule it onto a set of hardware resources. Scheduling the
execution on a single compute resource is trivial, but beyond that, the
scheduling problem is shown to be NP-Complete in many cases
\cite{Giersch-task-sharing-files-04}. Even when executed on modern day,
massively parallel compute architectures, these scientific applications can be
so large that they take days or months to execute.

\subsection{Cyberinfrastructure}

The focus of this project is scheduling a parameter sweep application onto a
cyberinfrastructure arranged in a \textit{master worker} structure *Figure2. The
master node is connected to $k$ worker nodes, $W_i$ where $1 \leq i \leq k$.
The links between master and worker nodes, $L_i$ where $1 \leq i \leq k$ have
a bandwidth of $BW_i$, denoted in megabytes per second (MBps).
Worker node $W_i$ has a compute speed $C_i$ denoted in floating point operations
per second (flops).

The master is responsible for assigning tasks to workers. Depending
on the application graph, before the master can map a task to a worker,
that tasks's required input files must be present at a persistant storage
service located on that same worker. Here we restrict the master from
sending more than one file at a time to a worker. Additionally the master
may start up a task at any moment on a worker if the required files are present
at the worker's persistant storage. Once a file has been sent from the master
to some worker, that file will be retained by the worker and may be used by
multiple tasks. Furthermore, the master may send the same file to multiple
workers such that multiple tasks that require the same file may be executed
in parallel by different workers.

Consider a simple application comprised of a single task $t$ that requires
some number of floating point operations to complete. $t$ also requires the set
of $F$ files. The master will synchronously send each file in $F$ to $W_1$
over $L_1$ then instruct $W_1$ to execute $t$. The expected makespan of this
application can be modeled by the following equation:make
$$ makespan_{expected} = \frac{\sum\limits_{f \in F}size(f)}{BW_1} + \frac{flops(t)} {C_1} $$
Modeling expected execution times is used by a number of scheduling heuristics
including MaxMin and MinMin.

\subsection{List Scheduling Heuristics}

Two of the most common online list scheduling algorithms are MaxMin and MinMin.
Both of these algorithms use estimated minimum completion times (MCT) of each
task to determine which set of files and tasks to map to what resource by
computing the following:

\begin{algorithm}[H]
  \caption{Estimate MCT}
  \begin{algorithmic}
    \STATE $estimated\_completion\_times = \emptyset$
    \STATE $tasks = pending\_tasks$
    \STATE $workers = all\_workers$
    \FOR{$t \in tasks$}
      \FOR{$w \in workers$}
        \STATE $ect = estimate\_completion\_time(t, w)$
        \STATE $estimated\_completion\_times.append((t, w, ect))$
      \ENDFOR
    \ENDFOR
  \end{algorithmic}
\end{algorithm}

\subsubsection{MaxMin}
first selects the task with the maximum of all the minimum estimated completion time.
Then, that task is assigned to the worker which will give it the largest estimated
completion
time out of all the possible workers it could have run on. This is to start long
tasks early on slower resources such that smaller tasks can be executed faster
on fast resources.

\subsubsection{MinMin}
selects the task, worker pair with the fastest estimated completion time and
assigns that task to that worker.


\section{genetic algorithm}
\label{sec:genetic_algorithm}

\subsection*{Chromosome Representation}
The chromosome representation of a schedule for this problem must be able to
capture two things: the order in which files are sent from the master and
the mapping of files to hosts. The order in which files are sent affect what
tasks are able to start at what times. Additionally, the mapping of files to
hosts dictates at which host a task can run. This information is encoded
with the following genotype:

$$[(f_1, h_1),(f_1,h_2),(f_2,h_1)...]$$

This genotype is specific to the application and cyberinfrastructure specifications.
For example, an application with input 10 files and a cyberinfrastructure with 10 workers
would yield a genotype with a 100 $(f,h)$ pairs. Because file transfers dictate
where a task may be executed, we ignore task, worker mappings because this can
ultimately be derived from file, worker mappings.

\subsection*{Fitness Evaluation}
The objective function is to minimize the application makespan (execution time),
therefore individual fitness is based soley on the application makespan given
its schedule. Ideally, it would be useful to execute this type of schedule
on a real platform, however obtaining the resources to do so may not be
possible for certain architectures (they may not exist or could very well be
too expensive). For that reason, we use WRENCH simulations to emulate aribtrary
applications and cyberinfrastructures. Given an individual's schedule as input
to the simulation, we can evaluate its fitness by obtaining the application
makespan using that schedule. The lower the application makespan, the higher
the individual's fitness.

\subsection*{Selection and Recombination}
Selection is done by taking the top 50\% of the population. Through recombination,
the remaining individuals are then replaced by offspring produced by the
top 50\%. Because the objective function is to minimize application makespan,
binary tournament selection is used twice to obtain two parents for crossover rather
than fitness proportional selection. Crossover is done using the following method:
Consider two parents $p1$, $p2$, the offspring $os$, a random crossover point $3$, and a schedule of
length $5$.

\begin{align}
 p1 & = [\mathcolorbox{lime}{(f_1, h_1), (f_2, h_2), (f_3, h_2)}, (f_3, h_1), (f_4, h_1)] \nonumber \\
 p2 & = [\mathcolorbox{orange}{(f_4, h_1}, (f_1, h_1),(f_3, h_2), \mathcolorbox{orange}{(f_3, h_1)}, (f_2, h_2)] \nonumber \\
 os & = [\mathcolorbox{lime}{(f_1, h_1), (f_2, h_2), (f_3, h_2)}, \mathcolorbox{orange}{(f_4, h_1), (f_3, h_1)}] \nonumber
\end{align}
The idea behind this method of crossover has been used in the development of
genetic algorithms for scheduling tasks on a multiprocessor machines \cite{wu-incremental-genetic-04}
where task ordering must be considered and therefore has been adopted here.
By using this method of crossover, we retain the order of file, worker
mappings for both of the parents in the offpsring.

\subsection*{Mutation}
If an individual is selected for mutation, the mutation operator simply swaps
two random file, worker mappings in the schedule. For example, say $p1$ from
the previous example is mutated. The result is $p1\_new$.

\begin{align}
 p1 & = [\mathcolorbox{lime}{(f_1, h_1)}, (f_2, h_2), (f_3, h_2), (f_3, h_1), \mathcolorbox{orange}{(f_4, h_1)}] \nonumber \\
 p1\_new & = [\mathcolorbox{orange}{(f_4, h_1)}, (f_2, h_2), (f_3, h_2), (f_3, h_1),\mathcolorbox{lime}{(f_1, h_1)}] \nonumber
\end{align}


\section{experimental details}
\label{sec:experimental_details}


\section{conclusion}
\label{sec:conclusion}


% For peer review papers, you can put extra information on the cover
% page as needed:
% \ifCLASSOPTIONpeerreview
% \begin{center} \bfseries EDICS Category: 3-BBND \end{center}
% \fi
%
% For peerreview papers, this IEEEtran command inserts a page break and
% creates the second title. It will be ignored for other modes.
%\IEEEpeerreviewmaketitle


%\section{Introduction}
% no \IEEEPARstart
%This demo file is intended to serve as a ``starter file''
%for IEEE conference papers produced under \LaTeX\ using
%IEEEtran.cls version 1.8b and later in~\cite{crypto_stega, stega}\cite{parallel_encrypt} \cite{lsbapproach}.
% You must have at least 2 lines in the paragraph with the drop letter
% (should never be an issue)
%I wish you the best of success.

%\hfill mds

%\hfill August 26, 2015

%\subsection{Subsection Heading Here}
%Subsection text here.


%\subsubsection{Subsubsection Heading Here}
%Subsubsection text here.


% An example of a floating figure using the graphicx package.
% Note that \label must occur AFTER (or within) \caption.
% For figures, \caption should occur after the \includegraphics.
% Note that IEEEtran v1.7 and later has special internal code that
% is designed to preserve the operation of \label within \caption
% even when the captionsoff option is in effect. However, because
% of issues like this, it may be the safest practice to put all your
% \label just after \caption rather than within \caption{}.
%
% Reminder: the "draftcls" or "draftclsnofoot", not "draft", class
% option should be used if it is desired that the figures are to be
% displayed while in draft mode.
%
%\begin{figure}[!t]
%\centering
%\includegraphics[width=2.5in]{myfigure}
% where an .eps filename suffix will be assumed under latex,
% and a .pdf suffix will be assumed for pdflatex; or what has been declared
% via \DeclareGraphicsExtensions.
%\caption{Simulation results for the network.}
%\label{fig_sim}
%\end{figure}

% Note that the IEEE typically puts floats only at the top, even when this
% results in a large percentage of a column being occupied by floats.


% An example of a double column floating figure using two subfigures.
% (The subfig.sty package must be loaded for this to work.)
% The subfigure \label commands are set within each subfloat command,
% and the \label for the overall figure must come after \caption.
% \hfil is used as a separator to get equal spacing.
% Watch out that the combined width of all the subfigures on a
% line do not exceed the text width or a line break will occur.
%
%\begin{figure*}[!t]
%\centering
%\subfloat[Case I]{\includegraphics[width=2.5in]{box}%
%\label{fig_first_case}}
%\hfil
%\subfloat[Case II]{\includegraphics[width=2.5in]{box}%
%\label{fig_second_case}}
%\caption{Simulation results for the network.}
%\label{fig_sim}
%\end{figure*}
%
% Note that often IEEE papers with subfigures do not employ subfigure
% captions (using the optional argument to \subfloat[]), but instead will
% reference/describe all of them (a), (b), etc., within the main caption.
% Be aware that for subfig.sty to generate the (a), (b), etc., subfigure
% labels, the optional argument to \subfloat must be present. If a
% subcaption is not desired, just leave its contents blank,
% e.g., \subfloat[].


% An example of a floating table. Note that, for IEEE style tables, the
% \caption command should come BEFORE the table and, given that table
% captions serve much like titles, are usually capitalized except for words
% such as a, an, and, as, at, but, by, for, in, nor, of, on, or, the, to
% and up, which are usually not capitalized unless they are the first or
% last word of the caption. Table text will default to \footnotesize as
% the IEEE normally uses this smaller font for tables.
% The \label must come after \caption as always.
%
%\begin{table}[!t]
%% increase table row spacing, adjust to taste
%\renewcommand{\arraystretch}{1.3}
% if using array.sty, it might be a good idea to tweak the value of
% \extrarowheight as needed to properly center the text within the cells
%\caption{An Example of a Table}
%\label{table_example}
%\centering
%% Some packages, such as MDW tools, offer better commands for making tables
%% than the plain LaTeX2e tabular which is used here.
%\begin{tabular}{|c||c|}
%\hline
%One & Two\\
%\hline
%Three & Four\\
%\hline
%\end{tabular}
%\end{table}


% Note that the IEEE does not put floats in the very first column
% - or typically anywhere on the first page for that matter. Also,
% in-text middle ("here") positioning is typically not used, but it
% is allowed and encouraged for Computer Society conferences (but
% not Computer Society journals). Most IEEE journals/conferences use
% top floats exclusively.
% Note that, LaTeX2e, unlike IEEE journals/conferences, places
% footnotes above bottom floats. This can be corrected via the
% \fnbelowfloat command of the stfloats package.




%\section{Conclusion}
%The conclusion goes here.




% conference papers do not normally have an appendix


% use section* for acknowledgment
%\section*{Acknowledgment}


%The authors would like to thank...





% trigger a \newpage just before the given reference
% number - used to balance the columns on the last page
% adjust value as needed - may need to be readjusted if
% the document is modified later
%\IEEEtriggeratref{8}
% The "triggered" command can be changed if desired:
%\IEEEtriggercmd{\enlargethispage{-5in}}

% references section

% can use a bibliography generated by BibTeX as a .bbl file
% BibTeX documentation can be easily obtained at:
% http://mirror.ctan.org/biblio/bibtex/contrib/doc/
% The IEEEtran BibTeX style support page is at:
% http://www.michaelshell.org/tex/ieeetran/bibtex/
\bibliographystyle{IEEEtran}
\bibliography{sources}
% argument is your BibTeX string definitions and bibliography database(s)
%
% <OR> manually copy in the resultant .bbl file
% set second argument of \begin to the number of references
% (used to reserve space for the reference number labels box)
%\begin{thebibliography}{1}

%\bibitem{IEEEhowto:kopka}
%H.~Kopka and P.~W. Daly, \emph{A Guide to \LaTeX}, 3rd~ed.\hskip 1em plus
%  0.5em minus 0.4em\relax Harlow, England: Addison-Wesley, 1999.

%\end{thebibliography}




% that's all folks
\end{document}
