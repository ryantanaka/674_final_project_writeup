\section{Introduction}

Distributed computing environments have become the platform of choice for
executing large scale scientific applications because they afford researchers
the ability to execute such applications in a fraction of the amount of time it
would take  to execute on a single machine. A "fraction of time" in this context
could mean an application makespan (execution time) of months instead of a year
assuming hardware resources are utilized in a clever manner.  This paper focuses
on one specific type of scientific application known as a \textit{parameter
sweep application}, where the entire application is composed of independent
tasks which can require a number of input files  \cite{Casanova-param-sweep-00},
\cite{Casanova-apples-param-sweep-00}. Tasks can require a varying amounts of
computation to complete and each task may use one or more files also used by
other tasks within the application. In order to execute a parameter sweep
application on a set of distributed computing resources, one must specify a
schedule with the following information: 1. when and where to send what file, 2.
when and where to execute what task. This decision making/scheduling process has
been proven to be NP-complete in many cases
\cite{Giersch-task-sharing-files-04}. Heuristic based, list scheduling
approaches such as \textit{Max-Min, Min-Min} are some efficient and effective ways
to schedule file transfers and assign tasks to compute hosts
\cite{Casanova-param-sweep-00},
\cite{Casanova-apples-param-sweep-00}, \cite{Giersch-task-sharing-files-04},
\cite{heuristics-99}.
Previous research has suggested that using genetic algorithms as a viable
meta-heuristic for the task scheduling problem \cite{wang-task-matching-97} ,
\cite{wu-incremental-genetic-04} where inter task dependencies exist. The
purpose of this project is to benchmark a schedule created by a genetic
algorithm (GA) and compare the schedule's performance with the afformentioned
\textit{Max-Min and Min-Min} heuristics. To accomplish this we use WRENCH, a
workflow management simulation framework \cite{casanova-works-2018}, to simulate
a parameter sweep application, the cyberinfrastructure it is to be executed on,
and the scheduling logic necessary to map portions of the application onto
compute and storage resources.

This paper is organized as follows. Section~\ref{sec:background} begins by
formally defining the problem of scheduling parameter sweep applications on
heterogenous resources. Then the Max-Min and Min-Min heuristics are introduced.
Section~\ref{sec:genetic_algorithm} covers the implementation details of the
genetic algorithm used. Section~\ref{sec:experimental_details} begins with a
description of the experimental scenario used to evaluate the GA against the
Max-Min and Min-Min heuristics followed by a discussion of the results.
