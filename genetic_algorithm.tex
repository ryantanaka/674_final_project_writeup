\section{genetic algorithm}
\label{sec:genetic_algorithm}

\subsection*{Chromosome Representation}
The chromosome representation of a schedule for this problem must be able to
capture two things: the mapping of files to hosts and the order in which files
are sent from the master. The mapping of files to hosts dictates at which host(s)
a task can run on. This mapping is described by the following relation which yields a set
of $nk$ $(file, worker)$ mappings.
\begin{align}
  domain & = \set{f_j \given 1 \leq j \leq n} \nonumber \\
  range  & = \set{w_i \given 1 \leq i \leq k} \nonumber
\end{align}
The order in which files are sent from the master affects when a task can
start as no task may start on a host which does not yet have the file(s) it requires.
Therefore, a list of $nk$ $(file, worker)$ mappings is used as the genotype such that the order
of each mapping determines the order of file transfers.
This genotype is specific to the application and cyberinfrastructure specifications.
For example, an application with 10 input files and a cyberinfrastructure with 10 workers
would yield a genotype with a 100 $(file, worker)$ pairs. One possible instance of a genotype
for this scenario may appear as follows:

$$[(f_1, h_1),(f_1,h_2),(f_2,h_1),(f_{10}, h_3)...]$$

Because file transfers dictate where a task may be executed, we ignore task,
worker mappings because this can ultimately be derived from file, worker
mappings (as well as from simulation the simulation data).

\subsection*{Fitness Evaluation}
The objective function is to minimize the application makespan (execution time),
therefore individual fitness is based soley on the application makespan given
its schedule. Ideally, it would be useful to execute this type of schedule
on a real platform, however obtaining the resources to do so may not be
possible for certain architectures (they may not exist or could very well be
too expensive). For that reason, we use WRENCH simulations to emulate aribtrary
applications and cyberinfrastructures. Given an individual's schedule as input
to the simulation, we can evaluate its fitness by obtaining the application
makespan using that schedule. The lower the application makespan, the higher
the individual's fitness.

In many cases, the application will complete (all tasks have been executed) before
all file transfers have completed. When evaluating an individual's fitness, the
application makespan is determined by the time at which all tasks complete, and
not the time at which all file transfers complete (every file worker mapping in the
genotype has been processed) because the remaining file transfers contribute
nothing to the application makespan.

\subsection*{Selection and Recombination}
Selection is done by taking the top 50\% of the population. Through
recombination, the remaining individuals are then replaced by offspring produced
by the top 50\%. Because the objective function is to minimize application
makespan, binary tournament selection is used twice to obtain two parents for
crossover rather than fitness proportional selection. Given a random cross over
point $k$, recombination is done by taking the first $k$ ordered pairs in the
first parent and copying them into the the first $k$ ordered pairs of the
offspring. The remaining ordered pairs that the offspring is missing is copied
from the second parent in the order that they appear there. For example,
consider two parents $p1$, $p2$, the offspring $os$, a random crossover point
$3$, and a schedule of length $5$.
\begin{align}
 p1 & = [\mathcolorbox{lime}{(f_1, h_1), (f_2, h_2), (f_3, h_2)}, (f_3, h_1), (f_4, h_1)] \nonumber \\
 p2 & = [\mathcolorbox{orange}{(f_4, h_1}, (f_1, h_1),(f_3, h_2), \mathcolorbox{orange}{(f_3, h_1)}, (f_2, h_2)] \nonumber \\
 os & = [\mathcolorbox{lime}{(f_1, h_1), (f_2, h_2), (f_3, h_2)}, \mathcolorbox{orange}{(f_4, h_1), (f_3, h_1)}] \nonumber
\end{align}
The idea behind this method of crossover has been used in the development of
genetic algorithms for scheduling tasks on a multiprocessor machines \cite{wu-incremental-genetic-04}
where task ordering must be considered and therefore has been adopted here.
By using this method of crossover, we retain the order of file, worker
mappings for both of the parents in the offpsring.

\subsection*{Mutation}
If an individual is selected for mutation, the mutation operator simply swaps
two random mappings in the schedule. For example, say $p1$ from
the previous example is mutated. The result is $p1\_new$.

\begin{align}
 p1 & = [\mathcolorbox{lime}{(f_1, h_1)}, (f_2, h_2), (f_3, h_2), (f_3, h_1), \mathcolorbox{orange}{(f_4, h_1)}] \nonumber \\
 p1\_new & = [\mathcolorbox{orange}{(f_4, h_1)}, (f_2, h_2), (f_3, h_2), (f_3, h_1),\mathcolorbox{lime}{(f_1, h_1)}] \nonumber
\end{align}
