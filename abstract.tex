\section{abstract}

% As a general rule, do not put math, special symbols or citations

\begin{abstract}

The optimization problem of scheduling tasks onto heterogenous resources in
distributed computing environments has been shown to be an NP-complete problem
in some cases and is the subject of ongoing research in the field of
distributed computing. A number of dynamic and static task-scheduling algorithms
have been proposed to tackle this problem. Of the static based scheduling
algorithms, there exist heuristic and random based search methods, all of which
can provide ideal task schedules in different application and hardware
configurations. The problem of accurately benchmarking these algorithms on
varying hardware configurations has proved to be time consuming for researchers,
and as a result, highly optimized simulation software has been developed over
the years to aid in the advancement of high performance and distributed
computing research. We first develop a genetic algorithm for the paremeter sweep
scheduling problem that uses that uses WRENCH, a workflow management
simulation framework, as a means of individual fitness evaluations. Then we
evaluate the performance of the schedule output by the genetic algorithm against
schedules produced by the Max-Min and Min-Min heuristics. Findings show that,
the heuristics, although simple to implement, provide good scheduling performance.
Additionally, the genetic algorithm's schedule performance approaches that of the
heuristics. However, 150 generations with a population size of 200 is not enough
to match the performance of those heuristics as movement through the search
space was slow, likely a result of the genotype implementation.


\end{abstract}
