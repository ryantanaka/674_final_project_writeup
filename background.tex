\section{background} \label{sec:background}

\subsection{Parameter Sweep Applications}

Scientific applications are often represented as directed acyclic graphs (DAGs)
where nodes represent computational tasks and input/output files. Links
generally will start at a file and extend to a task or vice versa. A link from a
file to a task denotes that the task requires the file as an input. Conversely,
a link from a task to a file signifies that the task creates that file as its
output. Here we focus on one such scientific application known as a parameter
sweep, where there are an independent set of files and tasks *Figure1. These
files have links to tasks, thus forming a bipartite graph. Depending on the
application, this bipartite graph may be arranged in a number of ways. For
example, consider an application with 10 files and 20 tasks. And say, each file
is used by a pair of tasks. This type of application may be easily parallelized
as the graph can be partitioned into sections. However, in the case of irregular
applications, the dependencies between files and tasks can make it difficult to
effectively schedule it onto a set of hardware resources. Scheduling the
execution on a single compute resource is trivial, but beyond that, the
scheduling problem is shown to be NP-Complete in many cases
\cite{Giersch-task-sharing-files-04}. Even when executed on modern day,
massively parallel compute architectures, these scientific applications can be
so large that they take days or months to execute.

\subsection{Cyberinfrastructure}

The focus of this project is scheduling a parameter sweep application onto a
cyberinfrastructure arranged in a \textit{master worker} structure *Figure2. The
master node is connected to $k$ worker nodes, $W_i$ where $1 \leq i \leq k$.
The links between master and worker nodes, $L_i$ where $1 \leq i \leq k$ have
a bandwidth of $BW_i$, denoted in megabytes per second (MBps).
Worker node $W_i$ has a compute speed $C_i$ denoted in floating point operations
per second (flops).

The master is responsible for assigning tasks to workers. Depending
on the application graph, before the master can map a task to a worker,
that tasks's required input files must be present at a persistant storage
service located on that same worker. Here we restrict the master from
sending more than one file at a time to a worker. Additionally the master
may start up a task at any moment on a worker if the required files are present
at the worker's persistant storage. Once a file has been sent from the master
to some worker, that file will be retained by the worker and may be used by
multiple tasks. Furthermore, the master may send the same file to multiple
workers such that multiple tasks that require the same file may be executed
in parallel by different workers.

Consider a simple application comprised of a single task $t$ that requires
some number of floating point operations to complete. $t$ also requires the set
of $F$ files. The master will synchronously send each file in $F$ then instruct
a worker to execute $t$. The expected makespan of this application can be
modeled by the following equation:

$$ makespan_{expected} = \frac{\sum\limits_{f \in F}size(f)}{BW_1} + \frac{flops(t)} {C_1} $$



\subsection{List Scheduling Heuristics} \subsubsection{MaxMin}
\subsubsection{MinMin}
